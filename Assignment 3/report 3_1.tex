\documentclass[12pt, a4paper]{report}
\usepackage{graphicx}
\usepackage[super]{nth}
\usepackage{color} % Colour control
\definecolor{db}{cmyk}{1,0.5,0,0.5}
\usepackage[Glenn]{fncychap}
\usepackage{enumerate}
\usepackage{diagbox}

\setlength{\parindent}{0pt}

\title{
    \vspace{0.5cm} \textcolor{db}{\textsc{LINGI2261: Artificial Intelligence}} \\
    \vspace{0.5 cm} \rule{10 cm}{0.5pt} \\
    \vspace{0.5 cm} \Large{Assignment 3: Project: Adversarial Search} \\
    \vspace{3 cm}
    \begin{flushright}
        \large
        \textbf{Groupe 81} \\
        Sanae \textsc{Abdelouassaa} \\
        Jiayue \textsc{Xue} \\     
    \end{flushright}
    \vspace{0.5 cm}
}
\begin{document}
\begin{figure}[t]
    \includegraphics[scale=1.0]{logo.png}
 \end{figure}   
\begin{figure}[t]
    \hspace{10 cm} \includegraphics[scale=0.5]{epl-logo.jpg}
\end{figure}

\maketitle

\tableofcontents

\chapter{Alpha-Beta search}

\section{Question 1:}
Perform the MiniMax algorithm on the tree in Figure 1, i.e. put a value to each node. Circle the move the root player should do.

\subsection{Answer:}
- See Figure \ref{fig1}: MiniMax .

   \begin{figure}[h]
      \caption{ MiniMax }
      \label{fig1}
         \includegraphics[scale=0.5]{MiniMax.png}
   \end{figure} 
   
\section{Question 2:}
Perform the Alpha-Beta algorithm on the tree in Figure 2. At each non terminal node, put the successive values of $\alpha$ and $\beta$. Cross out the arcs reaching non visited nodes. Assume a left-to-right node expansion.

\subsection{Answer:}
- See Figure \ref{fig2}: Alpha-Beta left to right expansion .

   \begin{figure}[h]
       \caption{Alpha-Beta left to right expansion}
       \label{fig2}
          \includegraphics[scale=0.5]{AlphaBeta.png}
    \end{figure}
    
\section{Question 3:}
Do the same, assuming a right-to-left node expansion instead (Figure3).

\subsection{Answer:}
- See Figure \ref{fig3}: Alpha-Beta right to left expansion .

    \begin{figure}[h]
        \caption{Alpha-Beta right to left expansion}
        \label{fig3}
         \includegraphics[scale=0.5]{Alphabeta1.png}
    \end{figure}
    
\section{Question 4:}
Can the nodes be ordered in such a way that Alpha-Beta pruning can cut off more branches (in a left-to-right node expansion)? If no, explain why; if yes, give the new ordering and the resulting new pruning.

\subsection{Answer:}
- Yes, there is another way using Alpha-Beta pruning with different ordered nodes.
\newline
- See Figure \ref{fig4}: Alpha-Beta right to left expansion with new ordering .
\newline
- The new ordered nodes are printed with a different color and we can see that we cut off one more branch than the ones cut off with the given tree.  

    \begin{figure}[h]
        \caption{Alpha-Beta right to left expansion with new ordering}
        \label{fig4}
         \includegraphics[scale=0.5]{q4.png}
    \end{figure}
    
\section{Question 5:}
How does Alpha-Beta need to be modified for games with more than two players?

\subsection{Answer:}
When playing a game with more than two players, we need to replace the single value for each node with a vector of values. For example, in a three-player game with players $A$, $B$, and $C$, a vector $<v_A, v_B, v_C>$ is associated with each node. 

For terminal states, this vector gives the utility of the state from each player’s viewpoint. 
For non-terminal states, each current player $p$ chooses the action among all the possible actions which maximizes the value corresponding to $p$ in the vector. 

Note that multi-player games usually involve alliances, whether formal or informal, among the players. 


\end{document}
 